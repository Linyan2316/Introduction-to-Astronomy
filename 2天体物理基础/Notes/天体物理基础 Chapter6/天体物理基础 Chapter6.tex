% !TEX encoding = UTF-8 Unicode
\documentclass[../天体物理基础.tex]{subfiles}
\begin{document}
\section{宇宙学}
\subsection{基本认知}
奥伯斯佯谬:宇宙无限,恒星分布均匀,那么夜晚天空应该像白天一样明亮。考虑球壳中恒星发光
\begin{equation}
E=\int_{0}^{\infty}n\times\frac{L}{4\pi r^{2}}\times4\pi r^{2}\mathrm{d}r\to\infty.
\end{equation}
其中$n$是恒星数密度,有几种解释:宇宙不是无限的;宇宙小尺度不是均匀的;宇宙年龄有限,遥远的星光还未到达我们;有星际消光存在;宇宙在膨胀光子发生红移。

只要物质和辐射之间有耦合,物质分布中的微小扰动都会被光子的扩散所牵制,发生粘滞性衰减。

物质和辐射解耦,物质变为独立后,质量稍有超出的区域,其范围就会由于对周围的引力作用而增长。

\subsection{观测宇宙学}
\subsubsection{观测手段}
我们可以根据球状星团年龄和放射性半衰期来确定宇宙年龄下限。

微波背景辐射CMB观测结果给出宇宙年龄$13.7\pm0.2\,\mathrm{Gyr}$。

物质含量可通过旋转曲线、星系团内部热气体辐射功率、引力透镜、近期轻元素丰度对原初核合成限制等确定。此外,物质、辐射、曲率、暗能量不同占比会影响CMB功率谱,也可作为限制。

对高红移\uppercase\expandafter{\romannumeral 1}a观测表明,它们的距离比预期的更高,因此宇宙正在加速膨胀,存在暗能量。

由于太阳的运动,CMB相对我们有一多普勒偶极不对称性,相对背景量级在$10^{-3}$。扣除此效应和星系前景辐射,CMB各向异性$\frac{\Delta{}T}{T}$仅有$10^{-5}$量级的变化。

这个各向异性来自声波振荡:引力和光压的相互作用导致等离子体在引力势中振荡,因此CMB功率谱携带了早期宇宙结构的信息。CMB中热点和冷点在天空中角距离尺度为$1^{\circ}$左右。

早期辐射主导,后期物质主导,现在暗能量主导。



\subsection{暴胀Inflation}
主要有两个问题。第一,我们认为波动的波长大于视界时就没有物理机制能破坏波动中蕴藏的信息。大一统时代的视界约$3\times10^{-26}\rm cm$,宇宙尺度$3\rm cm$,按理说宇宙这头和那头应该完全没关系了,CMB却高度各向同性,粒子如何达成信息交换?第二点,宇宙过于平直了。

所以我们认为,早期宇宙是很小的,物质耦合在一起,足以达到玻尔兹曼统计平衡。但是很快宇宙指数级膨胀,再高的山脉和地球比起来都微不足道,因此宇宙被拉平了。同时原先平衡的信息被拉到了视界尺度之外,一直留存了下来,保存到今天。

\subsection{暗物质与大尺度结构的形成}
Recombination前,辐射和正常(重子)物质是耦合在一起的。如果星系起源于宇宙早期正常物质的密度涨落,这种涨落也应该造成CMB的涨落。但我们知道CMB扰动仅有$10^{-5}$,如此密度不足以在已有的时间内形成星系和星系团。

因此,我们相信,暗物质的密度涨落应该在宇宙大尺度结构的形成中起主要作用。宇宙开始包含均匀分布的暗物质和正常物质。大爆炸后数千年暗物质开始成团。暗物质确定宇宙中物质的总体分布和大尺度结构。正常物质在引力作用下向高密度区域聚集,形成星系和星系团。

暗物质只参与弱作用和引力相互作用,与辐射场不耦合,因此暗物质的凝聚可以在辐射与正常物质脱耦前发生。(当然实际上有暗物质相互碰撞产生光子的模型,详情请见Chapter 3.4 in \textit{Modern Cosmology} by Scott Dodelson)暗物质的不均匀分布产生的引力变化导致微波背景辐射微小起伏。

暗物质的成分(根据退耦时刻粒子的能量与其静止质量相比较来区分):

热暗物质(HDM):粒子质量很小,速度接近光速(如中微子),影响大尺度结构。

冷暗物质(CDM):粒子质量较大、速度较慢,影响小尺度结构。

暗物质很可能同时包括热暗物质和冷暗物质。

是指铁的K层的电子占据的两条轨道中,其中一条有空位时,因L层的电子跃迁至K层而放出的发射线。$\mathrm{K\alpha}$线大致分为铁原子整体为中性的情况(中性铁$\mathrm{K\alpha}$线)和26个电子几乎全部电离的情况(高离化态铁$\mathrm{K\alpha}$线)。


\printbibliography

\end{document}