% !TEX encoding = UTF-8 Unicode
\documentclass[../天体物理基础.tex]{subfiles}
\begin{document}
\section{宇宙学}
\subsection{基本认知}
奥伯斯佯谬:宇宙无限,恒星分布均匀,那么夜晚天空应该像白天一样明亮。考虑球壳中恒星发光
\begin{equation}
E=\int_{0}^{\infty}\frac{L}{4\pi r^{2}}\times n\times4\pi r^{2}\mathrm{d}r\to\infty.
\end{equation}
其中$n$是恒星数密度,有几种解释:恒星寿命有限;宇宙不是无限的;宇宙小尺度不是均匀的;宇宙年龄有限,遥远的星光还未到达我们;有星际消光存在;宇宙在膨胀光子发生红移;宇宙膨胀过快,光子无法到达。

只要物质和辐射之间有耦合,物质分布中的微小扰动都会被光子的扩散所牵制,发生粘滞性衰减。

物质和辐射解耦,物质变为独立后,质量稍有超出的区域,其范围就会由于对周围的引力作用而增长。


宇宙充满物质,减速膨胀;没有物质,匀速膨胀

$\Omega_{0}>1$,束缚(封闭)宇宙,先膨胀后收缩

$\Omega_{0}<1$,开放宇宙,无限膨胀

$\Omega=1$,临界束缚,无限膨胀

\subsection{观测宇宙学}
\subsubsection{观测手段}
我们可以根据球状星团年龄和放射性半衰期来确定宇宙年龄下限。

微波背景辐射 CMB 观测结果给出宇宙年龄$13.7\pm0.2\,\unit{Gyr}$。

物质含量可通过旋转曲线、星系团内部热气体辐射功率、引力透镜、近期轻元素丰度对原初核合成限制等确定。此外,物质、辐射、曲率、暗能量不同占比会影响 CMB 功率谱,也可作为限制。

对高红移\uppercase\expandafter{\romannumeral 1}a 观测表明,它们的距离比预期的更高,因此宇宙正在加速膨胀,存在暗能量。

由于太阳的运动,CMB 相对我们有一多普勒偶极不对称性,相对背景量级在$10^{-3}$。扣除此效应和星系前景辐射,CMB 各向异性$\dfrac{\Delta{}T}{T}$仅有$10^{-5}$量级的变化。

这个各向异性来自声波振荡:引力和光压的相互作用导致等离子体在引力势中振荡,因此 CMB 功率谱携带了早期宇宙结构的信息。CMB 中热点和冷点在天空中角距离尺度为$\ang{1;;}$左右。

早期辐射主导,后期物质主导,现在暗能量主导。



\subsection{暴胀 Inflation}
主要有两个问题。第一,我们认为波动的波长大于视界时就没有物理机制能破坏波动中蕴藏的信息。大一统时代的视界约$\qty{3e-26}{cm}$,宇宙尺度$\qty{3}{cm}$,按理说宇宙这头和那头应该完全没关系了,CMB 却高度各向同性,粒子如何达成信息交换?第二点,宇宙过于平直了。

所以我们认为,早期宇宙是很小的,物质耦合在一起,足以达到玻尔兹曼统计平衡。但是很快宇宙指数级膨胀,再高的山脉和地球比起来都微不足道,因此宇宙被拉平了。同时原先平衡的信息被拉到了视界尺度之外,一直留存了下来,保存到今天。

$10^{-36}\,\unit{s}$宇宙高度对称。
$10^{-35}\,\unit{s},T<10^{28}\,\unit{K}$,强互作用分离,对称性破缺,宇宙真空有很高的能量。
$10^{-35}\text{\textendash}10^{-32}\,\unit{s}$,宇宙指数膨胀,尺度增大$10^{50}$倍

\subsection{暗物质与大尺度结构的形成}
Recombination 前,辐射和正常(重子)物质是耦合在一起的。如果星系起源于宇宙早期正常物质的密度涨落,这种涨落也应该造成 CMB 的涨落。但我们知道 CMB 扰动仅有$10^{-5}$,如此密度不足以在已有的时间内形成星系和星系团。

因此,我们相信,暗物质的密度涨落应该在宇宙大尺度结构的形成中起主要作用。宇宙开始包含均匀分布的暗物质和正常物质。大爆炸后数千年暗物质开始成团。暗物质确定宇宙中物质的总体分布和大尺度结构。正常物质在引力作用下向高密度区域聚集,形成星系和星系团。

不可能源于重子物质密度涨落

1. 脱耦前光子阻碍物质收缩

2. 涨落引起的变化在微波背景辐射仅表现为千分之一变化

3. 微波背景辐射记录的物质密度不足以在$100$亿年内形成星系

因此,暗物质起主要作用,它与辐射场不耦合,密度涨落也不会引起微波背景辐射变化


暗物质只参与弱作用和引力相互作用,与辐射场不耦合,因此暗物质的凝聚可以在辐射与正常物质脱耦前发生。(当然实际上有暗物质相互碰撞产生光子的模型,详情请见 Chapter 3.4 in \textit{Modern Cosmology} by Scott Dodelson)暗物质的不均匀分布产生的引力变化导致微波背景辐射微小起伏。

暗物质的成分(根据退耦时刻粒子的能量与其静止质量相比较来区分):

热暗物质(HDM):粒子质量很小,速度接近光速(如中微子),影响大尺度结构。

冷暗物质(CDM):粒子质量较大、速度较慢,影响小尺度结构。

暗物质很可能同时包括热暗物质和冷暗物质。

是指铁的 K 层的电子占据的两条轨道中,其中一条有空位时,因 L 层的电子跃迁至 K 层而放出的发射线。$\mathrm{K\alpha}$线大致分为铁原子整体为中性的情况(中性铁$\mathrm{K\alpha}$线)和 26 个电子几乎全部电离的情况(高电离化态铁$\mathrm{K\alpha}$线)。


\subsection{微波背景辐射}
早期炽热宇宙冷却,残留热辐射红移至微波波段

高度各向同性的$\qty{2.73}{K}$黑体辐射

偶极不对称性:太阳相对哈勃流运动引起的涨落,太阳朝向和背向温度变化约$10^{-3}$,$\Delta T=\qty{3.353}{mK}$

Sunyaev-Zel'dovich 效应:背景辐射穿过星系团部分光子被星际中气体的高能电子散射到高频波段,导致 CMB 能谱改变

扣除银河系尘埃辐射和偶极不对称性有十几万分之几$\rm K$变化,表明宇宙演化早期有微小的不均匀性,导致星系形成。

早期宇宙引力倾向于使重子成团,光子则倾向于消除各向异性。二者的相互作用导致等离子体在引力势中振荡。这种振荡产生的温度和密度变化被记录下来。宇宙膨胀,光子平均自由程增大,微波背景辐射小范围衰减,因此振动频谱记录了温度涨落和宇宙结构

宇宙学理论认为,CMB 的温度涨落遵循高斯分布,其功率峰值粗略对应于光子{}-{}重子流体脱耦时的振荡特征。峰值位置反映了宇宙曲率,重子密度,暗物质密度。

\printbibliography

\end{document}