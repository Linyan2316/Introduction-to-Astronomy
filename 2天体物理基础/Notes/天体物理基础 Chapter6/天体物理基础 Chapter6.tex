% !TEX encoding = UTF-8 Unicode
\documentclass[../天体物理基础.tex]{subfiles}
\begin{document}
\section{宇宙学}
\subsection{基本认知}
奥伯斯佯谬:宇宙无限,恒星分布均匀,那么夜晚天空应该像白天一样明亮。考虑球壳中恒星发光
\begin{equation}
E=\int_{0}^{\infty}\frac{L}{4\pi r^{2}}\times n\times4\pi r^{2}\symup{d}r\to\infty.
\end{equation}
其中$n$是恒星数密度,有几种解释:恒星寿命有限;宇宙不是无限的;宇宙小尺度不是均匀的;宇宙年龄有限,遥远的星光还未到达我们;有星际消光存在;宇宙在膨胀光子发生红移;宇宙膨胀过快,光子无法到达。

只要物质和辐射之间有耦合,物质分布中的微小扰动都会被光子的扩散所牵制,发生粘滞性衰减。

物质和辐射解耦,物质变为独立后,质量稍有超出的区域,其范围就会由于对周围的引力作用而增长。


宇宙充满物质,减速膨胀;没有物质,匀速膨胀

$\Omega_{0}>1$,束缚(封闭)宇宙,先膨胀后收缩

$\Omega_{0}<1$,开放宇宙,无限膨胀

$\Omega=1$,临界束缚,无限膨胀

\subsection{观测宇宙学}
\subsubsection{观测手段}
我们可以根据球状星团年龄和放射性半衰期来确定宇宙年龄下限。

微波背景辐射 CMB 观测结果给出宇宙年龄$13.7\pm\qty{0.2}{Gyr}$。

物质含量可通过旋转曲线、星系团内部热气体辐射功率、引力透镜、近期轻元素丰度对原初核合成限制等确定。此外,物质、辐射、曲率、暗能量不同占比会影响 CMB 功率谱,也可作为限制。

对高红移\uppercase\expandafter{\romannumeral 1}a 观测表明,它们的距离比预期的更高,因此宇宙正在加速膨胀,存在暗能量。

由于太阳的运动,CMB 相对我们有一多普勒偶极不对称性,相对背景量级在$10^{-3}$。扣除此效应和星系前景辐射,CMB 各向异性$\dfrac{\Delta{}T}{T}$仅有$10^{-5}$量级的变化。

这个各向异性来自声波振荡:引力和光压的相互作用导致等离子体在引力势中振荡,因此 CMB 功率谱携带了早期宇宙结构的信息。CMB 中热点和冷点在天空中角距离尺度为$\ang{1;;}$左右。

早期辐射主导,后期物质主导,现在暗能量主导。



\subsection{暴胀 Inflation}
主要有两个问题。第一,我们认为波动的波长大于视界时就没有物理机制能破坏波动中蕴藏的信息。大一统时代的视界约$\qty{3e-26}{cm}$,宇宙尺度$\qty{3}{cm}$,按理说宇宙这头和那头应该完全没关系了,CMB 却高度各向同性,粒子如何达成信息交换?第二点,宇宙过于平直了。

所以我们认为,早期宇宙是很小的,物质耦合在一起,足以达到玻尔兹曼统计平衡。但是很快宇宙指数级膨胀,再高的山脉和地球比起来都微不足道,因此宇宙被拉平了。同时原先平衡的信息被拉到了视界尺度之外,一直留存了下来,保存到今天。

$\qty{e-36}{s}$宇宙高度对称。
$\qty{e-35}{s},T<\qty{e28}{K}$,强互作用分离,对称性破缺,宇宙真空有很高的能量。
$10^{-35}\text{\textendash}\qty{e-32}{s}$,宇宙指数膨胀,尺度增大$10^{50}$倍

\subsection{暗物质与大尺度结构的形成}
Recombination 前,辐射和正常(重子)物质是耦合在一起的。如果星系起源于宇宙早期正常物质的密度涨落,这种涨落也应该造成 CMB 的涨落。但我们知道 CMB 扰动仅有$10^{-5}$,如此密度不足以在已有的时间内形成星系和星系团。

因此,我们相信,暗物质的密度涨落应该在宇宙大尺度结构的形成中起主要作用。宇宙开始包含均匀分布的暗物质和正常物质。大爆炸后数千年暗物质开始成团。暗物质确定宇宙中物质的总体分布和大尺度结构。正常物质在引力作用下向高密度区域聚集,形成星系和星系团。

不可能源于重子物质密度涨落

1. 脱耦前光子阻碍物质收缩

2. 涨落引起的变化在微波背景辐射仅表现为千分之一变化

3. 微波背景辐射记录的物质密度不足以在$100$亿年内形成星系

因此,暗物质起主要作用,它与辐射场不耦合,密度涨落也不会引起微波背景辐射变化


暗物质只参与弱作用和引力相互作用,与辐射场不耦合,因此暗物质的凝聚可以在辐射与正常物质脱耦前发生。(当然实际上有暗物质相互碰撞产生光子的模型,详情请见 Chapter 3.4 in \textit{Modern Cosmology} by Scott Dodelson)暗物质的不均匀分布产生的引力变化导致微波背景辐射微小起伏。

暗物质的成分(根据退耦时刻粒子的能量与其静止质量相比较来区分):

热暗物质(HDM):粒子质量很小,速度接近光速(如中微子),影响大尺度结构。

冷暗物质(CDM):粒子质量较大、速度较慢,影响小尺度结构。

暗物质很可能同时包括热暗物质和冷暗物质。

是指铁的 K 层的电子占据的两条轨道中,其中一条有空位时,因 L 层的电子跃迁至 K 层而放出的发射线。$\symup{K\alpha}$线大致分为铁原子整体为中性的情况(中性铁$\symup{K\alpha}$线)和 26 个电子几乎全部电离的情况(高电离化态铁$\symup{K\alpha}$线)。


\subsection{微波背景辐射}
早期炽热宇宙冷却,残留热辐射红移至微波波段

高度各向同性的$\qty{2.73}{K}$黑体辐射

偶极不对称性:太阳相对哈勃流运动引起的涨落,太阳朝向和背向温度变化约$10^{-3}$,$\Delta T=\qty{3.353}{mK}$

Sunyaev-Zel'dovich 效应:背景辐射穿过星系团部分光子被星际中气体的高能电子散射到高频波段,导致 CMB 能谱改变

扣除银河系尘埃辐射和偶极不对称性有十几万分之几$\rm K$变化,表明宇宙演化早期有微小的不均匀性,导致星系形成。

CMB 光子会被星系团散射。

早期宇宙引力倾向于使重子成团,光子则倾向于消除各向异性。二者的相互作用导致等离子体在引力势中振荡。这种振荡产生的温度和密度变化被记录下来。宇宙膨胀,光子平均自由程增大,微波背景辐射小范围衰减,因此振动频谱记录了温度涨落和宇宙结构

宇宙学理论认为,CMB 的温度涨落遵循高斯分布,其功率峰值粗略对应于光子{}-{}重子流体脱耦时的振荡特征。峰值位置反映了宇宙曲率,重子密度,暗物质密度。

宇宙线是十分高能的。高速粒子 ($\qty{e20}{eV}$) 与 CMB 光子反应,观测中会发现 Greisen-Zatsepin-Kuzmin cutoff.

\section{宇宙的基本观测事实}
大尺度满足空间平移不变性和旋转不变性,均匀且各向同性。

利用元素半衰期,利用球状星体 turnoff point 的年龄,利用白矮星光度函数的截止光度($L\propto{}Mt^{-\frac{7}{5}}$,更低光度宇宙年龄还不到)。

Sunyaev-Zel'dovich Effect: 星系团中的热气体与宇宙微波背景辐射的光子发生 ICS, 使得 CMB 谱向高能段偏移。如果星系团相对于 CMBR 静止参考系存在一个本动速度,被散射的 CMBR 光子会经历 Doppler 效应,也会导致 CMBR 辐射强度的变化。运动学 S-Z 效应比热 S-Z 效应弱。

星系团是宇宙中最大的引力束缚体系。其物质组成(重子物质和暗物质)应代表整个宇宙的组分。CMBR 光子经过星系团时产生的 S-Z 效应不依赖于星系团的红移。因此被视为强有力的宇宙学探针。

结合 X 射线观测确定 Hubble 常数和利用星系团的 S-Z 效应计数可以确定角直径距离、限制宇宙学基本参量、研究星系团的演化。从而最终限制宇宙学模型。

\section{宇宙膨胀的动力学}

解爱因斯坦方程得到 Friedmann 方程组
\begin{align}
\left(\frac{\dot a}{a}\right)^{2}&=\frac{8\pi\symup{G}}{3}\sum_{i}\rho_{i}-\frac{k}{a^{2}},\\
\frac{\ddot a}{a}&=-\frac{4\pi\symup{G}}{3}\sum_{i}\left(\rho_{i}+3\mathcal{P}_{i}\right).
\end{align}
对第一式求导,得到
\begin{equation}
2\dot a\ddot a=\frac{8\pi\symup{G}}{3}\left(\dot\rho a^{2}+2\rho\dot aa\right).
\end{equation}
结合第二式得到
\begin{equation}
\dot\rho+\frac{3\dot a}{a}\left(\rho+\mathcal{P}\right)=0.
\end{equation}

定义减速因子
\begin{equation}
q=-\frac{\ddot aa}{{\dot a}^{2}}=\frac{1}{2}\Omega_{\text{m}}+\Omega_{\text{r}}-\Omega_{\Lambda}.
\end{equation}

哈勃常数变化
\begin{align}
\Omega_{\text{r}}=\frac{8\pi\symup{G}}{3H_{0}^{2}}\rho_{\text{r}},\Omega_{\text{m}}=\frac{8\pi\symup{G}}{3H_{0}^{2}}\rho_{\text{m}},\Omega_{\Lambda}=\frac{8\pi\symup{G}}{3H_{0}^{2}}\rho_{\Lambda},\Omega_{\text{k}}=-\frac{k}{a^{2}H_{0}^{2}}.\\
H\left(z\right)=H_{0}\sqrt{\Omega_{\text{r}0}\left(1+z\right)^{4}+\Omega_{\text{m}0}\left(1+z\right)^{3}+\Omega_{\text{k}0}\left(1+z\right)^{2}+\Omega_{\Lambda0}}.
\end{align}

宇宙年龄
\begin{equation}
t_{0}=\int_{0}^{a_{0}}\frac{\symup{d}a}{\dot a}=\int_{0}^{\infty}\frac{\symup{d}z}{\left(1+z\right)H\left(z\right)}.
\end{equation}

视界
\begin{equation}
d_{\text{H}}=a_{0}\int_{0}^{t_{0}}\frac{\symup{c}\symup{d}t}{a\left(t\right)}.
\end{equation}
$z=1000$时视界的尺度是今天的视界尺度的$0.03$倍,按理说宇宙有$0.03^{-3}$个独立的区域。那么 CMB 各向同性是如何发生的?

宇宙起源于一个极小的区域(比经典大爆炸模型小),在暴胀前宇宙的大小远小于视界大小,因而具有相同的温度,暴胀后的宇宙依然具有相同的温度。

即使宇宙早期位形是高度弯曲的,经过暴胀会变为平直。

\section{CMB}
Acoustic oscillations:Recombination 以前声速达到$\dfrac{\symup{c}}{\sqrt{3}}$,直到共动声学世界在$R_{\text{s}}\sim150\,\symup{Mpc}$冻结。Recombination 时 Jeans 长度比视界尺度大得多,因此我们预计在所有亚视界尺度上都会有流体振荡,包括 Recombination 时刚好进入视界的尺度。视界尺度上流体的动力学将转化为宇宙微波背景辐射功率谱中的第一个声学峰。

曲率:声波震荡峰的位置。

暗能量:基本无影响。

重子含量:越多峰越高。

物质密度:物质越多,辐射 - 物质密度相等时刻越靠前,震荡变弱。

Silk damping tail:比自由程尺度小的扰动被抹平。

张量扰动能影响 CMB 偏振全部分量,标量扰动只能影响部分。

在大尺度结构的观测上,另一个重要的方向是重子声波振荡(Baryon Acoustic Oscillations)。在宇宙的光子退耦前,物质在引力的作用下向内收缩,而又在物质压强下向外扩张,从而形成了声波振荡,这种振荡会在光子退耦后保留下来。

\section{扰动}
扰动场$\delta{}\left(\vec{x}\right)$,Fourier 变换
\begin{align}
\delta{}\left(\vec{x}\right)&=\frac{1}{\left(2\pi\right)^{\frac32}}\int\delta{}\left(\vec{k}\right)e^{-i\symbf{k}\cdot\symbf{x}}\symup{d}^{3}\vec{k}.\\
\delta{}\left(\vec{k}\right)&=\frac{1}{\left(2\pi\right)^{\frac32}}\int\delta{}\left(\vec{x}\right)e^{i\symbf{k}\cdot\symbf{x}}\symup{d}^{3}\vec{x}.\\
\end{align}

定义物质扰动满足
\begin{align}
&\frac{\delta{}\rho}{\rho_{0}}=\delta{}\left(t\right)e^{i\symbf{k}\cdot\symbf{r}}.\\
&\frac{\symup{d}^{2}\delta{}}{\symup{d}t^{2}}+2\left(\frac{\dot a}{a}\right)\frac{\symup{d}\delta{}}{\symup{d}t}-\delta{}\left(4\pi\symup{G}\rho_{0}-k^{2}c_{\text{s}}^{2}\right)=0.
\end{align}

对于小扰动,忽略声波,考虑纯物质宇宙
\begin{equation}
4\pi\symup{G}\rho_{0}=\frac{2}{3t^{2}},\frac{\dot a}{a}=\frac{2}{3t},
\end{equation}
得到
\begin{equation}
\frac{\symup{d}^{2}\delta{}}{\symup{d}t^{2}}+\frac{4}{3t}\frac{\symup{d}\delta{}}{\symup{d}t}-\frac{2}{3t^{2}}\delta{}=0.
\end{equation}
取$\delta{}=at^{n}$,代入得到
\begin{equation}
n_{1}=\frac{2}{3},n_{2}=-1.
\end{equation}
由于宇宙膨胀,增长部分远慢于指数增长,$\delta{}\propto{}a$。

曲率为主的宇宙,
\begin{equation}
\frac{\symup{d}^{2}\delta{}}{\symup{d}t^{2}}+\frac{2}{t}\frac{\symup{d}\delta{}}{\symup{d}t}=0.
\end{equation}
解只有$n=0,-1$,即扰动不能增长。

辐射主导时,物质扰动不增长。

小尺度的扰动进入视界时间早,如果仍然为辐射为主,扰动基本不能增长。

大尺度的扰动在物质为主时进入视界,扰动随尺度因子线性增长。

%\printbibliography

\end{document}