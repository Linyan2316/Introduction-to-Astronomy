% !TEX encoding = UTF-8 Unicode
\documentclass[../天体物理基础.tex]{subfiles}
\begin{document}
\section{星系}
\paragraph{椭圆星系}~{}

星系中心最亮,亮度向边缘递减。我们可以用面亮度(单位平方角秒的星等)描述亮度,多数椭圆星系满足
\begin{equation}
I(R)=I_{0}\exp\,[-7.67(\frac{R}{R_{e}})^{\frac14}],
\end{equation}

\paragraph{漩涡星系}~{}


同样有面亮度分布,不过椭圆星系是$\exp\left(-\left(\dfrac{R}{R_{e}}\right)^{\frac14}\right)$,盘星系是$\exp\left(-\left(\frac{R}{R_{e}}\right)\right)$。


\printbibliography

\end{document}