% !TEX encoding = UTF-8 Unicode
\documentclass[../天体物理基础.tex]{subfiles}
\begin{document}
\section{星际介质与恒星形成}
分布在星际空间的物质,主要包括星际气体、星际尘埃、宇宙线和星际磁场。有趣的是,热气体、磁场、辐射场、宇宙线,它们的能量密度都是$\qty{1}{eV\cdot cm^{-3}}$.不过本文所提及的星系介质大约只包含星际气体和星际尘埃两种。

星云后面尘埃散射蓝光,看起来蓝。前面尘埃散射掉蓝光,因此星云看起来红且暗。

\subsection{星际气体和星际尘埃}
在银河系中,星际介质的质量约占银河系恒星质量的$10\%$,主要分布在距离银道面约$\qty{1000}{yr}$的范围内。它们的基本成分和主要属性如下:
\begin{table}[!htbp]
\centering
\caption{星际气体和星际尘埃的基本成分和主要属性}
\begin{tabular}{c c c}
\hline
性质 & 星际气体 & 星际尘埃\\
\cline{1-3}
质量百分比 & $99\%$ & $1\%$\\
\cline{1-3}
组成 & $70\%\ce{H},28\%\ce{He}$,其余$\ce{N,Ne,Na}$ & 冰、硅、石墨等固体粒子\\
\cline{1-3}
粒子数密度 & $10^{-4}\text{\textendash}\qty{e6}{cm^{-3}}$ & $\qty{e-13}{cm^{-3}}$\\
\cline{1-3}
质量密度 & $\qty{e-24}{g\cdot cm^{-3}}$ & $\qty{e-27}{g\cdot cm^{-3}}$\\
\cline{1-3}
温度 & $\qty{20}{K},\qty{100}{K},\qty{e4}{K},\ce{H2},\ce{H}\,\symup{\uppercase\expandafter{\romannumeral1}},\ce{H}\,\symup{\uppercase\expandafter{\romannumeral2}}$ & $10\text{\textendash}\qty{20}{K}$\\
\cline{1-3}
研究手段 & 星际吸收线、$\qty{21}{cm}$谱线,分子谱线 & 星际消光和红化、星际偏振、红外热辐射\\
\hline
\end{tabular}
\label{星际气体和星际尘埃的基本成分和主要属性。}
\end{table}

星际气体$n\in\left(10^{-3},10^{3}\right)\,\unit{cm^{-3}},T\in\left(10,10^{7}\right)\,\unit{K}$,因此星际气体的压强基本上是相同的。气体平均密度$\qty{1}{cm^{-3}}$,尘埃数密度$\qty{e-13}{cm^{-3}}$.

\subsubsection{星际气体概览}
气体为恒星形成提供原料,决定了星系的特性和星系演化。

星际气体的组成元素主要是氢元素。星际气体的空间分布不均匀,不同环境下 H 的存在方式不一样。
\begin{table}[!htbp]
\centering
\caption{不同成分的星际介质}
\begin{tabular}{c c c c c}
\hline
成分 & 观测证据 & 温度 $\left(\unit{K}\right)$ & 数密度$\left(\symup{cm^{-3}}\right)$ & 质量百分比\\
\cline{1-5}
分子云 & 红外辐射、紫外吸收线、$\ce{CO}$谱线 & $10\text{\textendash}50$ & $10^{2}\text{\textendash}10^{9}$ & $40\%$\\
\cline{1-5}
$\ce{H}\,\symup{\uppercase\expandafter{\romannumeral1}}$区 & 21 厘米谱线,紫外吸收线 & $50\text{\textendash}100$ & $1\text{\textendash}50$ & $40\%$\\
\cline{1-5}
$\ce{H}\,\symup{\uppercase\expandafter{\romannumeral2}}$区 & 光学和红外发射线、射电连续辐射 & $10^{4}$ & $10\text{\textendash}10^{4}$ & 极少\\
\cline{1-5}
云际气体 & 容许和禁戒发射线 & $7000\text{\textendash}10^{4}$ & $0.2\text{\textendash}0.3$ & $20\%$\\
\cline{1-5}
云际冕气 & X 射线辐射 & $10^{6}$ & $10^{4}\text{\textendash}10^{-3}$ & $0.1\%$\\
\hline
\end{tabular}
\label{不同成分的星际介质}
\end{table}

\subsubsection{电离氢气体}
发射星云:被高温恒星的紫外辐射电离的星际物质,也被称为 $\ce{H}\,\symup{\uppercase\expandafter{\romannumeral2}}$ 区。具有容许和禁戒发射线,颜色偏红,典型温度$\sim\qty{8000}{K}$.比如 O、B 型年轻恒星附近被紫外光子解离的气体形成的斯特龙根球。

电离产生大量自由电子,压力上升,电离氢区开始膨胀,而电离氢区的音速比中性氢高,形成一个激波波前。电离氢不断膨胀直到和中性氢达成压力平衡。在这样的区域中,星际气体的密度足以保持中性,但仍不足以阻止远紫外光的穿透,星际介质大部分区域是这样的光解离区 (PDR)。

\subsubsection{中性氢气体}
星际吸收线:星际气体中的原子受恒星紫外光子的电离而产生吸收线。由于星光可能穿过多块气体云,可能会出现多重吸收线。

星际气体低温$\to$窄吸收线

中性氢质量正比于 $\qty{21}{cm}$ 线发射亮度:
\begin{equation}
M_{\ce{H}\,\symup{\uppercase\expandafter{\romannumeral1}}}=2.3\times10^{5}\left(1+z\right)\left(\frac{D_{\text{L}}}{\unit{Mpc}}\right)^{2}\left(\frac{F}{\unit{Jy\cdot km\cdot s^{-1}}}\right)\,\unit{M_{\odot}}.
\end{equation}

\subsubsection{分子氢气体}
暗星云中心区域的射电观测无法探测到 21 厘米谱线。后来人们认识到暗星云主要由分子氢构成。

分子云是宇宙中最冷的东西,致密,中性,恒星形成的地点。目前已观测到约百种无机和有机分子。星系介质压力不足以克服氢原子间的相互作用,要形成分子氢需要其他物质作用:尘埃吸附氢原子并吸收能量,催化分子氢的形成。

示踪分子:氢分子不发射射电辐射,但是其他分子发射线大部分由氢分子的热运动碰撞激发产生。如利用$\ce{CO}$分子的$\qty{2.6}{mm}$射电辐射可以研究氢分子的分布。经验来说,$\ce{CO}$积分线密度$W$与分子氢柱密度成正比。

分子云:通过对$\ce{CO}$分子的观测,发现星际分子聚集成团而形成分子云,质量$\qty{e6}{M_{\odot}}$,直径$\qty{600}{ly}$,密度$10^{3}\text{\textendash}\qty{e5}{cm^{-3}}$.

分子云占据银盘内大约$1\%$的空间,质量大约占星际气体总质量的$50\%$.银河系中$M_{\ce{H}}\sim\left(4\text{\textendash}8\right)\times\qty{e9}{M_{\odot}},M_{\ce{H2}}=\dfrac{1}{2}M_{\ce{H}}$.

巨分子云:质量$\qty{e6}{M_{\odot}}$,直径$\qty{300}{ly}$,温度$\qty{20}{K}$,数密度$\qty{e3}{}\text{\textendash}\qty{e5}{cm^{-3}}$,寿命$10^{7}\text{\textendash}\qty{e8}{yr}$。大约$10\%$的分子云足够致密,可以形成恒星。

\subsubsection{气体循环}
\begin{figure}[!htbp]
\centering
\includegraphics[width=14cm]{figures/figure3_1.png}
\captionsetup{justification=raggedright, singlelinecheck=false}
\caption{星际气体的物理状态和物质循环。}
\label{星际气体的物理状态和物质循环。}
\end{figure}

漩涡星系中,致密分子气体、尘埃、年轻恒星主要集中在星系的旋臂区域。银河系中,分子氢主要分布在$R<R_{0}$区域,中性氢分布范围更高更广(甚至比恒星更延伸),但星系中心数百秒差距内没有分子氢和中性氢。

在星际云间的空间中也存在气体。主要有中性的稀薄气体和更稀薄的热气体$\left(\qty{e4}{K}\right)$。

紫外和 X 射线观测还发现存在一类温度高达$10^{6}\text{\textendash}\qty{e7}{K}$的热气体,称为云际冕气 (coronal gas). 20\text{\textendash}60\%的星际空间被云际冕气占据。这些气体的高温主要来自超新星的加热。

电离氢在超新星爆发作用下可形成日冕气体。恒星星风和 SN 爆发也会产生日冕气体。日冕气体光致冷却成电离氢,电离氢复合成原子氢,原子氢在尘埃上形成稀疏的分子氢,分子云坍缩形成致密分子氢,开始恒星形成。

\subsubsection{星际尘埃}
尘埃主要成分为硅或石墨颗粒,外面被冰或二氧化碳包裹。星际尘埃质量的 10\text{\textendash}20\%可能在最小的粒子中,如多环芳香烃。

尘埃尺寸$0.001\text{\textendash}\qty{1}{\mu m}$,星系介质一半的金属在尘埃中。尘埃在 RGB、AGB 包层和超新星中形成,在激波和高热环境中被摧毁。凝结于红巨星大气的尘埃会在在恒星演化晚期被吹向星际空间。尘埃和气体成团状分布,高银纬处可假设尘埃量与气体量成正比。

尘埃通过吸收和瑞利散射引发消光。尘埃能有效地散射和吸收波长小于其自身尺度的辐射。它们吸收光学和紫外波段的星光,在红外波段辐射。大尘粒,$T\sim\qty{30}{K}$,辐射峰值$\qty{100}{\mu m}$. $T>\qty{100}{K}$的尘埃,其峰值$\sim\qty{30}{\mu m}$.小尘埃粒子辐射短于$\qty{30}{\mu m}$.

星际尘埃对星光的散射截面随波长的变化而不同
\begin{align}
\sigma_{\lambda}\propto{}\frac{a^{3}}{\lambda},&\lambda\ge a,\\
\sigma_{\lambda}\propto{}a^{2},&\lambda\ll a.
\end{align}
因此尘埃对蓝光吸收和散射较多而对红光散射较少,导致星际消光和红化。剩下的星光大致满足
\begin{equation}
\log E\left(\lambda\right)\sim\frac{1}{\lambda}.
\end{equation}

星际尘埃提供了原子聚集形成分子的场所,并屏蔽了星光中的紫外线使分子免遭瓦解。尘埃还有催化剂的作用。

星际尘埃的观测:

光学观测:反射星云和暗星云

反射星云:星云通过尘埃反射附近的热星的光而发光,颜色偏蓝

暗星云:大量尘埃阻挡了星云内部或恒星后面的星光

红外观测:尘埃的热辐射。尘埃粒子受宇宙线和附近热星辐射的加热,温度可以达到$\qty{100}{K}$,产生红外热辐射。

偏振观测:星光偏振现象反映了尘埃呈长条形定向排布。

\subsubsection{Heating and Cooling}
Heating 的关键在于给粒子提供动能。主要有超新星爆发(热气体、激波、包层碰撞)、辐射(紫外光子、来自尘埃的电子)、宇宙线。

Cooling 的关键则是通过辐射转移动能,因此需要满足三点:系统在某些波段光学薄,非常依赖特定的温度,数密度越高越好 ($\propto{}n^{2}$).主要有轫致辐射 ($>\unit{keV}$)、重元素内层电子 ($>\qty{100}{eV}$)、电子能级 ($\sim\unit{eV}$)、分子转动能级 ($\unit{meV}$).

加热效率$\Gamma$主要和位置、电离能级、尘埃组分有关,冷却效率主要和温度和数密度有关。光学薄系统中,平衡满足
\begin{equation}
\mathcal{L}=\Lambda-\Gamma=0.
\end{equation}

\begin{figure}[!htp]
\centering
\tikzset{every picture/.style={line width=0.75pt}}
\begin{tikzpicture}[x=0.5pt,y=0.5pt,yscale=1,xscale=1]
\draw (0,0) -- (0,250) ;
\draw [shift={(0,250)},rotate=270][line width=0.75](10.93,-3.29) .. controls (6.95,-1.4) and (3.31,-0.3) .. (0,0) .. controls (3.31,0.3) and (6.95,1.4) .. (10.93,3.29); 
\draw (0,0) -- (350,0) ;
\draw [shift={(350,0)},rotate=180][line width=0.75](10.93,-3.29) .. controls (6.95,-1.4) and (3.31,-0.3) .. (0,0) .. controls (3.31,0.3) and (6.95,1.4) .. (10.93,3.29); 
\draw (20,220) -- (60,180) ;
\draw (60,180) -- (120,180) ;
\draw (120,180) -- (130,120) ;
\draw (130,120) -- (190,120) ;
\draw (190,120) -- (200,60) ;
\draw (200,60) -- (290,40) ;
%Text
\draw (-20,300) node [anchor=north west][inner sep=0.75pt]   [align=left] {$\log T$};
\draw (350,-20) node [anchor=north west][inner sep=0.75pt]   [align=left] {$\log n$};
\draw (20,60) node [anchor=north west][inner sep=0.75pt]   [align=left] {Heating,$\mathcal{L}<0$};
\draw (20,260) node [anchor=north west][inner sep=0.75pt]   [align=left] {轫致辐射};
\draw (140,200) node [anchor=north west][inner sep=0.75pt]   [align=left] {重离子};
\draw (200,140) node [anchor=north west][inner sep=0.75pt]   [align=left] {电子能级};
\draw (290,40) node [anchor=north west][inner sep=0.75pt]   [align=left] {分子振动};
\end{tikzpicture}
\captionsetup{justification=raggedright, singlelinecheck=false}
\caption{加热冷却图。水平平台和能级有关,因为 cooling 只在特定能级发生,在这些能级处温度变化一点,要维持冷却效率不变,就要引起数密度巨大的变化。同时水平平台稳定平衡,竖直平台不稳定平衡。}
\label{加热冷却图。}
\end{figure}

\subsubsection{单区瞬时演化模型}
我们已经知晓,恒星的形成与演化会不断将 $\ce{H},\ce{^{}He}$ 元素转化为金属元素,如果我们通过观测大致获得星系产生重元素的平均速率,那么就可以通过建立星际介质中金属丰度和时间的函数关系。

为简化模型,我们假设星系气体总是充分混合,恒星聚变产物瞬时返回星际介质,没有任何气体从星系逃离也没有任何气体流入。记时刻$t$的气体质量为$M_{\text{g}}\left(t\right)$,金属丰度为$Z\left(t\right)$, 低质量恒星(寿命太长,产物在可预见的时标内不会返回到星际空间中)和大质量恒星遗迹(中子星、黑洞等,重元素被“锁”在其中也无法返回星际介质)的质量为$M_{*}\left(t\right)$,从恒星中返回星际介质的气体金属丰度为$p$.

现在有$\Delta{}M_{\text{g}}'\left(t\right)$的气体形成恒星,同时有$\Delta{}M_{\text{g}}''\left(t\right)$的气体返回星际介质,因此$\Delta{}M_{\text{g}}\left(t\right)=\Delta{}M_{\text{g}}''\left(t\right)-\Delta{}M_{\text{g}}'\left(t\right)$,同时$\Delta{}M_{*}\left(t\right)=-\Delta{}M_{\text{g}}\left(t\right)$的气体留在恒星遗迹中。星际介质金属丰度的变化为
\begin{equation}
\Delta{}Z=Z'-Z\approx\frac{ZM_{\text{g}}-Z\Delta{}M_{\text{g}}'+p\Delta{}M_{\text{g}}''}{M_{\text{g}}}-Z=\frac{p\Delta{}M_{\text{g}}}{M_{\text{g}}}.
\end{equation}

如果$p$不依赖于金属丰度,那么
\begin{equation}
\frac{\Delta{}Z}{p}=-\frac{\Delta{}M_{\text{g}}}{M_{\text{g}}}.
\end{equation}
因此气体金属丰度随时间增加
\begin{equation}
Z\left(t\right)=Z_{0}+p\ln\left[\frac{M_{\text{g0}}}{M_{\text{g}}}\right].
\end{equation}
$M_{\text{g}0}-M_{\text{g}}$是已经转化成恒星的气体质量,因此金属丰度小于一定值$Z$的恒星质量为
\begin{equation}
M\left(<Z\right)=M_{\text{g0}}\left(1-\symup{e}^{-\frac{Z-Z_{0}}{p}}\right).
\end{equation}
金属丰度在$Z$和$Z+\Delta{}Z$之间恒星的质量为
\begin{equation}
\frac{\symup{d}M_{*}}{\symup{d}Z}\Delta{}Z\propto{}\exp\left\{-\left[Z-Z_{0}/p\right]\Delta{}Z\right\}.
\end{equation}

银河系核球区域,核球引力可能成功地留住了所有气体,并完全将其变成了恒星,所以观测结果能够与单区瞬时循环模型预言复合较好。太阳附近模型检验则有一些问题。模型给出贫金属星$Z<\qty{0.25}{Z_{\odot}}$比例约 40\%,实际上仅有$25\%$.可能出于以前恒星形成所导致的重元素预增丰,或者闭合假定和瞬时混合假定不成立,气体内流外流影响了重元素丰度。

\subsection{恒星形成}

\subsubsection{简单分析}

银河系内恒星总质量$\qty{5e10}{M_{\odot}}$,年龄$\qty{e10}{yr}$,因此银河系平均恒星诞生率$\qty{5}{M_{\odot}yr^{-1}}$.

O 型星寿命约$\qty{e6}{yr}$,是最近形成的天体,观测 O 型星确定目前的恒星诞生率为$\qty{1.65}{M_{\odot}yr^{-1}}$.

康德{}-{}拉普拉斯星云说:
太阳系起源于旋转的星云,由于冷却凝缩,星云旋转速度加快,呈扁平状,当离心力超过引力时逐渐分裂出许多环状物。星云中心部分凝聚成太阳,各个环状物碎裂并凝结成围绕太阳运行的行星。

后来人们逐渐认识到,恒星形成与分子云的引力坍缩有关。

星云质量足够高时,引力超过热运动提供的压力,就会引起星云坍缩。极限质量被称为金斯质量。动能$\symup{K}=\dfrac{3}{2}Nk_{\text{B}}T$,势能$\symup{U}\sim-\dfrac{3}{5}\dfrac{\symup{G}M^{2}}{R},2\symup{K}<\left\vert{}\symup{U}\right\vert{}$时星云坍缩。

金斯判据只适用于均匀分布的气体。实际上气体的状态和环境十分复杂,需要考虑的因素包括星系产生的潮汐力,分子云的转动、湍动和磁场,分子云的形态等等。

位力定律分析:

无粘性流体的运动方程为:
\begin{align}
\rho\frac{\symup{d}\symbf{u}}{\symup{d}t}&=-\nabla{}P-\rho\nabla\phi_{g}+\frac{1}{\symup{c}}\symbf{J}\times\symbf{B}.\\
\rho\frac{\symup{d}\symbf{u}}{\symup{d}t}&=-\nabla P-\rho\nabla\phi_{g}+\frac{1}{4\pi}\left(\symbf{B}\cdot\nabla \symbf{B}\right)-\frac{1}{8\pi}\nabla{}\left\vert{}\symbf{B}\right\vert{}^{2}.
\end{align}
不考虑外界压强,即自引力主导下,方程转化为
\begin{equation}
\frac{1}{2}\frac{\partial{}^{2}I}{\partial{}t^{2}}=2T+2U+W+M,
\end{equation}
其中转动惯量$I=\int\rho\left\vert{}\symbf{r}\right\vert{}^{2}\symup{d}^{3}x$,动能$T=\dfrac{1}{2}\int\rho\left\vert{}\symbf{u}\right\vert{}^2\symup{d}^{3}x$,热能$U=\dfrac{3}{2}\int nk_{\text{B}}T\symup{d}^3x$,引力势能$W=\dfrac{1}{2}\int\rho\phi_{g}\symup{d}^{3}x$,磁能$M=\dfrac{1}{8\pi}\int\left\vert{}\symbf{B}\right\vert{}^{2}\symup{d}^{3}x$.

气体云在引力坍缩时,$\dfrac{1}{2}\dfrac{\partial{}^{2}I}{\partial{}t^{2}}\approx-\dfrac{\symup{G}M^{2}}{R}$.坍缩时标
\begin{equation}
t_{\text{ff}}\sim\left(\frac{1}{4\pi{}\symup{G}\rho}\right)^{\frac{1}{2}}\sim5\times10^{5}\left(\frac{n}{\qty{e4}{cm^{-3}}}\right)^{-\frac{1}{2}}\,\unit{yr}.
\end{equation}

星云坍缩触发机制:

1. 激波压缩:超新星爆发、热星辐射、银河系旋臂转动等过程产生激波,激波压缩周围星云,使其密度增大,触发恒星形成,其过程类似链式反应。

2. 星云与旋臂区域碰撞,由于旋臂密度更高,星云坍缩,产生恒星。

恒星形成的 Kennicutt-Schmidt 定律
\begin{figure}[!htbp]
\centering
\includegraphics[width=13cm]{figures/figure3_2.png}
\captionsetup{justification=raggedright, singlelinecheck=false}
\caption{恒星形成的 Kennicutt-Schmidt 定律}
\label{恒星形成的 Kennicutt-Schmidt 定律}
\end{figure}

初始质量函数 (Initial Mass Function): 单位体积内形成的恒星的相对数目在质量上的分布,可以表示成
\begin{equation}
\xi\left(m\right)\symup{d}m=\xi_{0}m^{-\alpha}\symup{d}m,
\end{equation}
其中$m=\dfrac{M}{\unit{M_{\odot}}}$. E.Salpeter 最早提出初始质量函数的概念,并发现$\alpha=2.35$.

\subsubsection{恒星形成理论}

质量越大的恒星,演化到主序的时间越短,主序上的位置越高

\paragraph{低质量恒星的形成}~{}

1. 分子云和云核

最初大体处于流体静力学平衡,分子云缓慢旋转和收缩,引力能转化为动能进而转化为内能,产生辐射。由于云核光学薄,热量可以不受阻碍地逃逸,云核等温坍缩。

由于旋转速度加快,此时分子云可以分裂成更小的云核,云核进一步收缩和分裂,导致密度上升,金斯质量下降。核心逐渐变得不透明,趋向绝热坍缩,温度迅速上升,金斯质量增大,云核停止分裂,开始坍缩。

2. 云核引力坍缩

坍缩时标
\begin{equation}
\frac{\symup{d}^{2}r}{\symup{d}t^{2}}=-\frac{\symup{G}M_{r}}{r^{2}}\to t_{\text{ff}}=\left(\frac{3\pi{}}{32\symup{G}\rho}\right)^{\frac{1}{2}}.
\end{equation}

密度均匀的星云同步坍缩,中心致密星云自内向外坍缩

由于角动量大小的差异,中心区域的气体直接落入引力势阱中形成原恒星,而外层气体形成一个围绕中心区域旋转的扁平吸积盘。

3. 原恒星的吸积与成长

原恒星通过吸积盘快速吸积气体。转动和磁场产生的喷流成功带走吸积盘气体角动量,进而吸积盘可以吸附质量。内部完全对流,赫罗图上沿林中四郎线演化

4. 主序前星的形成与演化

吸积率下降,外向流张角变大,原恒星质量不再实质增长,成为主序前星,但内部温度还未升高到$\ce{H}$的点火温度。原恒星以热时标收缩。中心星的辐射和星风驱散部分吸积物质,其余部分形成原行星盘。

5. 零龄主序

核心的热核反应开始进行,光度约为目前太阳光度的三分之二

\paragraph{大质量恒星形成的问题}~{}

观测方面,新生的大质量恒星深埋在致密的云核中,观测时标短。此外,紫外辐射对环境的破坏导致很难追溯初始形成条件。理论方面,恒星的辐射加热、电离周围气体会阻碍其坍缩和吸积。

模型有三种,

1. 云核直接坍缩

小质量恒星形成模型的放大版本,$M_{\text{J}}>\qty{10}{M_{\odot}}$,辐射加热或湍动、磁场主导内部压力。但是湍动作用未知,尚未观测到大质量恒星前核。

2. 竞争吸积

$M_{\text{J}}\sim\qty{0.5}{M_{\odot}}$,处于引力优势位置的原恒星更容易通过吸积周围的气体生长,但这样提供的恒星形成效率低。

3. 小质量原恒星的并合,要求原恒星中心数密度达到$\qty{e8}{pc^{-3}}$.

\paragraph{极小质量恒星可能的形成机制}~{}

湍动分子云核的分裂

多个原恒星胚胎相互作用时被抛出的较小的个体

原恒星盘的碎裂

原恒星核的光致腐蚀

\subsubsection{其他问题}

形成恒星的云核的角动量比恒星大几个量级,说明角动量转移到多星系统的轨道角动量或通过吸积盘向外转移

星云的转动坍缩往往伴随多星系统的形成

转动星云在坍缩过程中会形成围绕原恒星旋转的环或盘。




O、B 型恒星表面温度高能产生大量紫外 ($\lambda<\qty{912}{\angstrom}$) 光子电离氢,产生$\ce{H}\,\symup{\uppercase\expandafter{\romannumeral2}}$区。如斯特龙根球就是年轻 O、B 型恒星周围存在的电离氢区。光致电离产生的大量自由电子之间相互频繁碰撞,建立电子气的平衡态速度分布。



% \printbibliography

\end{document}