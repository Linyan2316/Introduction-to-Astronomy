% !TEX encoding = UTF-8 Unicode
\documentclass[../天体物理基础.tex]{subfiles}
\begin{document}
\section{AGN}

$\left[\ce{O}\,\symup{\uppercase\expandafter{\romannumeral3}}\right]$禁线可作为 AGN 吸积过程中,产生的 outflow 的示踪谱线。

铁的 K 层的电子占据的两条轨道中,其中一条有空位时,L 层的电子跃迁至 K 层会放出发射线。$\symup{K\alpha}$线大致分为铁原子整体为中性的情况(中性铁$\symup{K\alpha}$线)和 26 个电子几乎全部电离的情况(高电离化态铁$\symup{K\alpha}$线)。

正常星系辐射来自恒星的累积辐射,观测上表现为黑体辐射,峰值能量位于光学波段。

活动星系光度高,连续谱为热辐射和非热辐射,峰值能量位于远红外波段;强发射线 + 偏振辐射;短时间内大幅光变,说明中心能源源大小不超过一光年;往往有着特殊形态,包括亮核、喷流、不规则形态。

AGN 能产生 X 射线(逆康普顿散射),紫外光子(为主,来源于吸积盘落入黑洞),尘埃红外热辐射,和射电(强大的喷流)。总光度可达$10^{44}\text{\textendash}\qty{e47}{erg\cdot{}s^{-1}}$,但发光范围仅有十余个天文单位。

20\text{\textendash}40\% 的活动星系有核活动。大部分活动星系是椭圆星系。

距离我们越远,星系活动性越频繁。活动星系高光度表明它们的演化时标很短,只是星系演化中的一个阶段。大约 1\% 的星系有明显的活动星系核。可能所有星系都经历过 AGN 阶段。

\subsection{常见的 AGN}
\subsubsection{Seyfert Galaxies}
几乎都是漩涡星系。具有亮核和宽发射线。

光谱中有非常强的容许和禁戒发射线。

根据光谱特征可分为\uppercase\expandafter{\romannumeral1}型和\uppercase\expandafter{\romannumeral2}型,\uppercase\expandafter{\romannumeral1}型 Seyfert 星系中允线宽度达$\qty{e4}{km\cdot{}s^{-1}}$,来自黑洞和吸积盘附近致密 ($n\sim\qty{e9}{cm^{-3}}$) 的高电离气体,远超禁戒线宽度,\uppercase\expandafter{\romannumeral2}型二者都相当窄,$\le\qty{e3}{km\cdot{}s^{-1}}$,主要来自禁线 ($n\sim\qty{e3}{cm^{-3}}$),主要是星系中被吸积盘照亮的气体。

\uppercase\expandafter{\romannumeral1}和\uppercase\expandafter{\romannumeral2}型 AGN 是同一种,前者盘面垂直视线方向,后者平行因此宽线发射被环挡住了。证据是\uppercase\expandafter{\romannumeral2}型 AGN 的一些宽线发射呈现出汤姆孙散射导致的偏振。

\subsubsection{Radio Galaxies}

绝大部分活动星系是射电星系。射电光度远超正常星系

大部分是椭圆星系,也是星系团中最大最明亮的星系

辐射能量集中在射电波段,射电辐射主要来自于射电瓣

非热同步加速谱

\subsubsection{Quasar}

光学形态与恒星类似,历史上指两类天体,一种是 Quasi-stellar radio sources, 一种是 Quasi-stellar objects, 当下用类星体一词统称二者。

有强且宽的未知发射线。类星体是人们观测到的最遥远、最古老也是辐射功率最大的河外天体。


类星体是活动星系核,位于星系团中,宿主星系中恒星辐射形成弥漫结构

\subsubsection{Blazar}

宿主星系都是巨椭圆星系。由光学激变源和蝎虎 BL 天体组成。光学激变源都是强射电源。蝎虎 BL 天体是弱射电源,非热连续谱,发射线很弱。

光变快速且剧烈

\subsection{统一模型}
高光度、非热连续辐射、快速光变、特殊形态(亮核、喷流)、宽发射线(气体高速运动)

超大质量黑洞吸积过程中的引力能释放。

吸积气体在黑洞周围形成吸积盘,在螺旋接近黑洞的过程中受到加热,释放巨大的能量。

在吸积过程中可能形成双极喷流,喷流在远离核区处与星系际物质相互作用形成射电瓣

一般认为赛弗特星系的允许线产生于离星系核较近的致密气体(宽线区,broad line region, BLR),禁戒线产生于相对较远的稀薄气体(窄线区,narrow line region, NLR)

统一模型示意图:Beckmann \& Shrader (2012)

\begin{equation}
L_{\nu}=\int_{\nu_{1}}^{\nu_{2}}F_{\nu}\symup{d}\nu=\int_{\nu_{1}}^{\nu_{2}}\nu F_{\nu}\frac{\symup{d}\nu}{\nu}=\int_{\nu_{1}}^{\nu_{2}}\nu F_{\nu}\symup{d}\ln\nu,
\end{equation}

决定活动星系观测特性的因素可能包括吸积盘倾角和黑洞吸积的爱丁顿比率。

黑洞存在的证据:

核区的恒星与气体运动、核区的巨 Maser,反响映射 (Reverberation mapping),高分辨率观测确定核区大小,天体运动确定核区质量,质量/空间尺度比确定极高密度天体(黑洞)

反响映射:黑洞的连续辐射电离了宽线区内的气体,产生发射线,连续辐射与发射线的光变有时间延迟$t$,由此确定宽线区大小,由宽线区发射线宽度可以得到气体运动速度,进而确定质量。

有很多方法可以测量黑洞的质量。可以测量恒星和气体的动力学。反响映射。吸积盘会旋转,发出的光会受引力红移和 beaming effect 影响。引力波。

$M$-$\sigma$关系:黑洞与核球的共同演化?

宇宙诞生后约 10 亿年,星系开始形成。富气体星系并合,气体向核区流动,激发了星暴活动和超大质量黑洞的增长,但这一阶段类星体往往被尘埃所阻挡。一旦黑洞主导了核区的能量过程,其反馈活动驱散了气体和尘埃,核区就会出现一个明亮的类星体。

feedback 会驱散或加热气体,抑制/终止恒星形成过程。

黑洞周围气体逐渐耗尽,核心光度逐渐减小,热辐射越来越重要。

气体被进一步加热和驱散,核心活动无法维持,辐射开始主要来自于热辐射,表现为正常星系。

通过研究星系团对背景类星体或星系产生的引力透镜现象,可以得到星系团内的(暗)物质分布和宇宙大尺度结构的信息。

\subsection{吸积盘}
假设半径$r$处高度为$z$,平行角动量方向引力压力平衡
\begin{align}
\frac{\partial{}p}{\partial{}z}&=\frac{\symup{G}M\rho}{r^{2}}\frac{z}{r},\\
p&=\frac{\symup{G}M\rho}{r^{2}}\frac{h^{2}}{r}.\\
\frac{h}{r}&=\frac{c_{\text{s}}}{\sqrt\dfrac{\symup{G}M}{r}}.
\end{align}

旋转时会产生一个粘滞力(单位面积)
\begin{equation}
f=\eta r\frac{\partial{}}{\partial{}r}\left(\frac{\sqrt\dfrac{\symup{G}M}{r}}{r}\right)=\eta r\frac{\partial{}\Omega}{\partial{}r}.
\end{equation}
力矩
\begin{equation}
T=fAr=2\pi \eta hr^{3}\frac{\partial{}\Omega}{\partial{}r}.
\end{equation}
净力矩
\begin{equation}
M=\frac{\partial{}T}{\partial{}r}\symup{d}r.
\end{equation}
对于开普勒盘,$\Omega\propto{}r^{-\frac32}$,净扭矩为负,因此环被外部物质减慢,质量向内流动,角动量向外转移。

能量释放比较简单。考虑质量吸积率
\begin{equation}
\dot m=2\pi rv_{r}\Sigma,
\end{equation}
能量释放引力势能的一半,
\begin{equation}
\symup{d}L=\dfrac{1}{2}\frac{\symup{G}M}{r^{2}}\dot m\symup{d}r=4\pi r\symup{d}r\sigma T^{4}.
\end{equation}
因此温度
\begin{equation}
T=\left(\frac{\symup{G}M\dot m}{8\pi\sigma}\right)^{\frac14}r^{-\frac34}.
\end{equation}
看似质量越大温度越高,不过质量越大半径也会越高,超大质量黑洞附近温度反而低,因为$T\sim M^{-\frac{1}{2}}$。一个太阳质量的中子星温度$\qty{e8}{K}$能有 X 光子,$\qty{e7}{M_{\odot}}$黑洞可能只有$\qty{e5}{K}$,紫外光子。

总光度为
\begin{equation}
L=\dfrac{1}{2}\symup{G}M\dot m\left(\frac{1}{r_{\min}}-\frac{1}{r_{\max}}\right).
\end{equation}
对于没有自旋的黑洞而言,
\begin{equation}
r_{\min}=3R_{\text{s}}=\frac{6\symup{G}M}{\symup{c}^{2}},L=\frac{1}{12}\dot m\symup{c}^{2}.
\end{equation}
对于一般的黑洞光度大约满足$L=0.057\dot m\symup{c}^{2}$。原子核聚变能量利用率不过$0.7\%$,黑洞吸积能量释放是它的十倍。如果黑洞有自转,效率可以达到$0.4$,因为最小稳定轨道离黑洞更近。

吸积盘是一个个圆轨道叠加组成的,总辐射强度也可看作一个个黑体谱叠加,大致满足
\begin{equation}
I_{\nu}\propto{}v^{\frac{1}{3}}\int_{r_{\min}}^{r_{\max}}x^{\frac{4}{3}}\left[\exp\left(\frac{h\nu}{k_{\text{B}}T}\right)-1\right]^{-1}x^{\frac13}\symup{d}x.
\end{equation}

Eddington 光度$L=\dfrac{4\pi\symup{c}\symup{G}Mm_{\ce{p}}}{\sigma_{\text{T}}}$给定了最大吸积速率。
\begin{equation}
\frac{\dot M}{M}=\frac{4\pi\symup{c}\symup{G}m_{\ce{p}}}{\varepsilon_{\text{L}}\sigma_{\text{T}}\symup{c}^{2}}.
\end{equation}
由此给定了黑洞吸积时标
\begin{equation}
\tau_{\text{s}}=\varepsilon_{\text{L}}\frac{\sigma_{\text{T}}\symup{c}^{2}}{4\pi\symup{c}\symup{G}m_{\ce{p}}}.
\end{equation}

%\printbibliography

\end{document}