% TeX Live 2025 XeLaTeX
\documentclass[UTF8, 12pt, a4paper, oneside, onecolumn, fontset=none]{ctexart}
\usepackage{anyfontsize}
\usepackage{geometry}
\geometry{left=1.5cm,right=1.5cm,top=1.8cm,bottom=0.9cm}

\usepackage{unicode-math, mathrsfs, xcolor, inputenc}
\definecolor{pagecolor}{HTML}{C7EDCC}
\pagecolor{pagecolor}
\setCJKmainfont[AutoFakeBold, AutoFakeSlant]{LXGW WenKai}
\setmainfont[AutoFakeBold, AutoFakeSlant]{LXGW WenKai}
\setmathfont{Latin Modern Math}
\setmathrm{Latin Modern Math}

\usepackage{amsmath, amsthm, cases, graphicx, subfigure, tikz, caption, subcaption, siunitx, keyval, physics2}
\DeclareSIUnit\angstrom{Å}
\usephysicsmodule{ab,ab.braket}
\usepackage[version=4]{mhchem}
\usepackage[flushleft]{threeparttable}
\usepackage{hyperref, bookmark}
\hypersetup{colorlinks=true, urlcolor=blue, filecolor=blue, linkcolor=blue}
\usepackage[backend=biber,sorting=none]{biblatex}
\addbibresource{/Users/Linyan/Zotero/MyBibTeX.bib}

\ctexset{section={format=\bfseries\zihao{4}\flushleft}}
\renewcommand\thesubsection{\thesection.\arabic{subsection}}
\numberwithin{equation}{subsection}
\renewcommand\theequation{\thesubsection.\arabic{equation}}
\makeatletter
\renewcommand{\maketag@@@}[1]{\hbox{\m@th\normalsize\normalfont#1}}%
\makeatother
\usepackage{indentfirst}

\usepackage{listings}
\lstset{
       basicstyle      =    \zihao{-5}\ttfamily,
       numberstyle     =    \zihao{-5}\ttfamily,
       keywordstyle    =    \color{blue},
       keywordstyle    =    [2]\color{teal},
       stringstyle     =    \color{magenta},
       commentstyle    =    \color{red}\ttfamily,
       breaklines=true,     % 自动换行
       columns=fixed,       % 固定字间距
       flexiblecolumns,     % 别问为什么,加上这个
       numbers         =    left,% 行号的位置在左边
       showspaces      =    false,% 是否显示空格
       numberstyle     =    \zihao{-5}\ttfamily,% 行号的样式,小五号,tt 等宽字体
       showstringspaces=    false,
       captionpos      =    t,%这段代码的名字所呈现的位置,t 指的是 top 上面
       frame           =    lrtb,% 显示边框
}

\lstdefinestyle{Python}{
       language=Python,     
       basicstyle      =    \zihao{-5}\ttfamily,
       numberstyle     =    \zihao{-5}\ttfamily,
       keywordstyle    =    \color{blue},
       keywordstyle    =    [2]\color{teal},
       stringstyle     =    \color{magenta},
       commentstyle    =    \color{red}\ttfamily,
       breaklines      =    true,% 自动换行,建议不要写太长的行
       columns         =    fixed,% 如果不加这一句,字间距就不固定
       basewidth       =    0.5em,
}

\title{实测天体物理}
\author{林衍}
\date{\today}

\begin{document}
\maketitle
\subsection*{前言}


\newpage
\tableofcontents

\newpage
\section{太阳}
黑子由本影与半影组成,温度约$3000\text{\textendash}\qty{4500}{K}$, 产生原因是黑子下方的强磁场$\left(\sim\qty{e3}{G}\right)$阻碍了物质对流,使热量难以到达该处。

太阳内部等离子体转动产生磁场。由于对流区较差转动产生横向磁场。磁力线突出日面就形成黑子、日珥和耀斑。

太阳磁场的发电机理论:太阳内部等离子体的转动产生磁场:
\begin{equation}
\frac{\partial{}\symbf{B}}{\partial{}t}=\nabla\times\left(\symbf{u}\times\symbf{B}-\eta\nabla\times\symbf{B}\right),
\end{equation}
其中$\eta$是磁扩散系数。

太阳对流区具有较差转动的特点,一般认为环向磁场产生于辐射区和对流区的交界面差旋层。
\begin{figure}[!htbp]
\centering
\begin{minipage}[t]{0.45\textwidth}
\includegraphics[width=6cm]{figures/figure1_1.png}
\end{minipage}
\begin{minipage}[t]{0.45\textwidth}
\includegraphics[width=6cm]{figures/figure1_2.png}
\end{minipage}
\captionsetup{justification=raggedright, singlelinecheck=false}
\caption{太阳磁力线的运动。}
\label{太阳磁力线的运动。}
\end{figure}

赤道式优点:天体的视运动可以很容易地利用赤经轴的匀速转动来补偿,视场的星像位置没有相对转动,在观测条件最好的天顶位置没有盲区。

赤道式缺点:承载量有限,望远镜的口径有限制,非对称式结构望远镜口径的极限是 $\qty{2.5}{m}$ 左右,对称式结构望远镜的口径一般不超过 $\qty{5}{m}$.

地平式优点:镜筒只在一个方向上承受弯矩的作用,承载重量大;作用在望远镜支臂上的力是不变的竖直向上的力,对望远镜的指向没有影响,回转半径小,可以使用小尺寸的圆顶或更为紧凑、跟随望远镜转动的观察室。望远镜的安装地点与当地的地理纬度无关。

地平式缺点:天顶盲区,在这个区域无法对天体进行跟踪观察,像场旋转。

黄道光 (Zodiacal Light): 位于地球上低纬度和中纬度地带的人于春季黄昏后在西方地平线上,或于秋季黎明前在东方地平线上所见到的淡弱的三角形光锥。黄道光沿着黄道向上伸展,可达地平线以上$\ang{30;;}$左右。

黄道光的可见时间不长。春季黄昏后见到的黄道光,随着夜幕完全降临就逐渐消逝;秋季黎明前见到的黄道光,随着东方逐渐吐白就隐没于晨曦之中。

黄道光很暗弱,必须在良好的环境条件下才能有效地观测。春季黄昏后和秋季黎明前黄道面的空间方向恰好最接近于垂直地平面,所以这时黄道光就升得较高,容易看到。

在黄道上黄道光向太阳一直可以延伸到太阳的近旁,而在背太阳方向其亮度不断下降,然而离太阳 $\ang{135;;}$ 左右,亮度反而有所增加,并在离太阳 $\ang{180;;}$ 处又达到极大值,这就是对日照。

对日照的亮度更比黄道光暗弱,但与周围背景相比,却明显地明亮些。由于它十分微弱,几乎任何人为的光亮都足以使它相形见拙而无法观测。

对日照的成因一般倾向于是行星尘埃物质反射太阳光而形成的。

\subsection{恒星基本参数测定}
\subsubsection{半径}

直接测量:

空间望远镜成像

地面望远镜干涉,光学望远镜的分辨率可以达到毫角秒的分辨率。

掩食法

2.1.6 质量

双星 (binary stars) 由在彼此引力作用下以椭圆轨道互相绕转的两颗恒星组成的双星系统。

银河系大部分的恒星(尤其是大质量恒星)位于双星和聚星系统中。

验证万有引力定律

测量恒星质量

研究恒星结构(形状、大小、大气)

研究(特殊条件下的)恒星演化

研究物质交流和吸积过程

利用开普勒第三定律
\begin{equation}
\frac{\alpha^3}{T^2}=\frac{\symup{G}}{4\pi^2}(M_1+M_2)
\end{equation}
将恒星质量测量转换为双星轨道半长径和周期测量,进而转换成对恒星位置和运动的测量、或者测量谱线频率变化,光度变化和轨道倾角。还需知道一颗恒星的质量和质量比。

目视双星 即望远镜分辨

天体测量双星 双星的一颗子星较暗无法观测到,但通过较亮子星的自行轨迹的变化可以推测其伴星的存在。

分光双星 谱线多普勒位移确定速度和倾角,这里可以加视向速度变化曲线 

食双星 相互交食 光变曲线 反映了温度比、轨道倾角和恒星的大小


\subsection{光学望远镜}
衍射极限导致光在望远镜中心成像为一个光斑$\theta=1.22\dfrac{\lambda}{D}$,其中$D$是望远镜口径(那个圆筒的有效直径),这影响了光学望远镜的分辨率(光斑叠在一起就分不清了)。

望远镜的优点:

1. 集光能力强,故可观测暗弱天体。

2. 空间分辨本领强$(\theta=1.22\lambda/D)$,可看清天体的表面细节。

3. 追踪天体的周日视运动,从而可长时间曝光。

4. 拓展观测波段。

光学望远镜可分为光学系统、机械结构和控制系统。

\subsubsection{表征性能的物理量}
口径:未被镜框挡住的那部分物镜的直径$D$。因此口径越大,就能看见越暗弱的天体,所能看到的天体越多,所需曝光时间越短。

相对口径:$A=D/f$。倒数为焦比。面源成像面积正比于$f^{2}$,光通量$\propto{}D^{2}$,因此单位面积光强正比于$A^{2}$。相对口径 A 越大,就越能看到更暗弱的面源天体。实际的 A 值往往受到像差的限制而不能随意取值。折射$A\sim\dfrac{1}{7}\text{\textendash}\dfrac{1}{5}$,反射$\sim\frac{1}{5}\text{\textendash}\frac{1}{3}$,折反射$\sim\dfrac{1}{3}\text{\textendash}\frac{1}{2}$,甚至可以大于 1.

目视望远镜的放大率指的是角放大率,$G=\dfrac{\tan\omega'}{\tan\omega}=\dfrac{f_{\text{物}}}{f_{\text{目}}}$.

照相望远镜没有目镜,直接在焦平面(不是像平面)成像,放大率指的是底片比例尺。

分辨角:两天体的像刚刚能被分辨开的时候的角距离。倒数为分辨本领。

视场:能被望远镜良好成像的天空区域的角直径。照相望远镜的视场就是物镜的视场,可以用底片比例尺求,也可以直接取位于像质良好区域两端的两个星像的角距离。一般$A$越大,视场越大。

极限星等(贯穿本领):晴朗无月的夜间,用望远镜观看或拍摄天顶附近的 A$0$型恒星所能达到的最暗星等。和辐射探测器、台址有关,但是和口径的关系最密切。

\subsubsection{光学望远镜的像差(补充内容)}
由于衍射效应,即使恒星发出的光是单色光,在物镜焦平面上的恒星像也会成为由明暗交替的圆环包围着中央亮斑的衍射图样。

从几何学的角度看,即使不考虑衍射,对于实际光学系统而言,也只有在以下条件下,高斯的理想光学系统理论才能适用:

1. 光以极窄的近轴光束入射到光学系统,以至于和光轴夹角的正弦和正切可以用角度本身来表示。2. 无色散。

实际上这两个条件往往无法都满足,致使点源可能成为形状很不规则的斑点,面源的像会发生形变,这都是像差的表现。实际操作中只要将光学系统的像差减小到和衍射效应没有明显区别即可。

不满足条件 1,会产生球差、彗差、像散、场曲、畸变。

不满足条件 2,会产生色差。

球差:远轴光学并不汇聚于焦点。反射镜是球面镜尤为严重,为此中大型反射望远镜不采用球面镜。

彗差:宽阔光束以一定角度入射成像会得到彗状斑点。齐明条件:入射光线在入射光瞳处的高度$y$等于焦距$f$乘折射(或反射)光线与光轴的夹角的正弦$\sin\omega'$。

像散:点光源成像是个椭圆,原因是球形波面通过实际光学系统后变成非球形波面。

场曲:像散在面光源成像中的反映,它使得面光源的成像平面变成曲面。即使像散消除,场曲现象仍然可能存在。在天文工作中,消除场曲的方法是:将底片形状弯曲成与望远镜焦面形状相符;加改正镜。

畸变:成像面上放大率或底片比例尺不是常数而产生的物与像的相似性被破坏的一种像差。例如,在畸变的影响下,正方形的物体将成像为枕形的或桶形的,前者称为枕形畸变,后者称为桶形畸变。

位置色差:轴向光束经物镜折射后不同波长的光成像在不同位置的一种像差现象。可以使用复消色差透镜进行消除。

倍率色差:当包含各种波长的斜光束成像时,不同波长的光相交焦平面上不同的点,即其放大率不一样,结果导致轴外物点将成像为一短光谱,这种色差称为倍率色差。当对实际光学系统消除位置色差时,倍率色差也会相应减小。

球差$\propto{}D^{3}$,彗差$\propto{}\omega D^{2}$,像散(场曲)$\propto{}\omega^{2}D$,畸变$\propto{}\omega^{3}$,色差。

\subsubsection{光学系统分类}
\paragraph{折射望远镜}~{}

伽利略:物镜是凸透镜,目镜是凹透镜。

开普勒:两片凸透镜,放大倍数有了显著提升。

大口径光学透镜制作困难,且极易发生形变。玻璃会吸收入射光,尤其是对蓝、紫光。六种像差很严重。

\paragraph{反射望远镜}~{}

看看图就好。

简单式物镜系统(牛顿、主焦点):只有一块曲面反射镜。

复杂式物镜系统(卡塞格林、R-C、耐焦、折轴):两块及以上曲面反射镜。

牛顿:凹球面反射镜,物镜主焦点前加上一块与光轴成$\ang{45;;}$的平面反射镜。仅消除了色差。为了消除球差,一般中型和大型的反射望远镜的物镜都是旋转抛物面镜。抛物面镜虽无色差、球差,但对轴外光束却存在严重的彗差,从而大大限制了望远镜的视场。

对牛顿式反射望远镜而言,由于终端设备不在入射光路中,不会挡光,因此可以安置较大的终端设备。有时在牛顿式系统镜筒侧面安置不同的终端设备,转动平面镜就可以使其中一个处于工作状态。

主焦点:由一块反射镜组成的系统称为主焦点系统。这种系统常用于口径较大的反射望远镜中,其相对口径通常在$\dfrac{1}{5}\text{\textendash}\dfrac{1}{2.5}$之间。由抛物面反射镜构成的主焦点系统没有球差,但有严重的彗差。主焦点式的反射望远镜视场特别小,适于做单颗星的分光、测光工作。由于只经过一个反射面,由反射引起的光的损失小,所以它还适于做小视场的暗弱天体的照相工作。

\begin{figure}[!htbp]
\centering
\includegraphics[width=10cm]{figures/figure1_4.png}
\captionsetup{justification=raggedright, singlelinecheck=false}
\caption{}
\label{}
\end{figure}


卡塞格林:最常使用的天文望远镜系统。主镜抛物面,副镜凸双曲面,二者重合的焦点是主焦点,另一个焦点是卡焦。根据圆锥曲线的光学性质,原来无球差地会聚到 F1 点的光线经双曲面反射后会无球差地会聚到 F2 点。卡镜的等效焦距由下式决定:
\begin{equation}
f'=f_{\text{\text{抛}}}\frac{\text{双曲面到 F1}}{\text{双曲面到 F2}}.
\end{equation}

在卡塞格林系统焦点上安置终端设备,并不像主焦点系统那样存在挡光问题,因而可以安置较大的终端设备,便于作天体物理工作。为了在卡塞格林式望远镜上同时安装几种不同的终端设备,常在卡塞格林焦点前加上一块和光轴成$\ang{45;;}$的平面镜,使成像于镜筒侧面(类似牛顿系统),这种卡塞格林系统称为耐司姆斯系统。当平面镜改变方向,可形成一圈范围的焦点,就可使位于不同位置的终端设备处于工作状态。

R-C 系统:主镜十分接近于凹双曲面,副镜十分接近于凸双曲面,这种系统无色差、球差、彗差,且满足齐明条件。其光路图与卡塞格林系统相似,因而常被看成满足齐明条件的卡塞格林系统。

R-C 系统视场比同样相对口径的主焦点系统和卡塞格林系统大好几倍,但像场是个曲面,要采用弯曲底片以消除场曲。我国 $\qty{2.16}{m}$ 望远镜就是采用 R-C 系统。

R-C 系统的缺点是,因为其主镜是双曲面,所以去掉副镜,直接在主焦点处工作时是有球差的,必须在焦点前加像场改正透镜。

折轴系统:大型反射望远镜经常设有折轴系统。使用曲率半径更高的副镜增加焦距,使用反射镜将焦点转移到固定位置。一般用于安装大型仪器(如高分辨率光谱仪)或者进行多台仪器光干涉。折轴焦点也叫 Coude 焦点,放置大型仪器的房间也叫 Coude 房。

\subsubsection{辐射探测器}
主要依靠 CCD,量子效率(探测器所能记录的光子数除以同样条件下一个理想探测器所能记录的光子数)高,可达 75\%,照相底片才 5\%。观测结果可数字化,方便后续工作。

\subsubsection{主动光学}
通过改变主镜的形状调整因重力形变、风速起伏等因素造成的像质扭曲,这些修正部件工作由计算机控制,在相对较低的频率下工作,这种系统称为主动光学系统。

\subsubsection{自适应光学}
Adaptive optics,使用可变形镜面矫正因大气抖动造成光波波前发生的畸变,从而改进光学系统性能。自适应光学系统必须通过分析有限的数据在每一毫秒内做出新的修正。

\subsubsection{射电望远镜}
可以全天候观测,受地球大气和星系物质影响较小。通过干涉原理,可以把多个射电望远镜连在一起当一个大射电望远镜使用,空间分辨率取决于间距(我们称其为天线基线 baseline)。

\subsubsection{空间望远镜}
一个是 X 射线能量高,会穿透镜面,要采用掠射望远镜(入射光子几乎平行镜面)。这就产生一个问题,镜面只有一个小圆环能采光,于是我们多造几个套在一起,像玫瑰花一样采光。

\subsection{地球大气}
课件只讲了大气窗口(仅对射电和光学透明)和视宁度(seeing)。以下是补充内容。

\textbf{大气消光}:大气分子和固体颗粒对辐射的吸收和散射作用。它既减弱了天体辐射的强度,又改变了天体辐射的能谱(颜色)。

X 射线和伽马射线:与电离层内带电离子作用,根本观测不到。

紫外波段:$\ce{N2}$产生吸收带,主要分布于$2000\,\unit{\angstrom}$直到远紫外区$1000\,\unit{\angstrom}$波段范围。$\ce{O2}$吸收带位于$2300\text{\textendash}2900\,\unit{\angstrom}$。$\ce{O3}$同样影响红外和紫外观测。

可见光波段:主要是大气的散射作用,但不厉害。

红外波段:$\ce{CO2}$对红外辐射吸收强,还影响$\unit{mm}\text{\textendash}\unit{cm}$波,很大高度内均影响观测。$\ce{H2O}$影响红外和$\unit{mm}$波观测,$\qty{12}{cm}$以上可忽略。$\ce{CH4,CO}$吸收、影响红外和微波观测,尘埃影响红外和红外短波长观测。

射电波段:电离层使大于 23.5 米的电磁辐射全反射,影响射电观测。

因此观测中常要进行消光改正。因为密度随高度的指数变化,下层大气的消光作用是主要的。
首先定义大气质量:在天文学上指来自天体的光穿过大气层的路径长度。
\begin{equation}
M\left(z\right)=\frac{\tau_{\lambda}\left(z\right)}{\tau_{\lambda}\left(0\right)}.
\end{equation}

定义透射系数$P=\symup{e}^{-\tau}$,天顶距为$z$的方向,透射系数为
\begin{equation}
P_{\lambda}\left(z\right)=\symup{e}^{-\tau_{\lambda}\left(z\right)}=e^{-\tau_{\lambda}\left(0\right)M^{\left(z\right)}}=\left(P_{\lambda}\left(0\right)\right)^{M\left(z\right)}.
\end{equation}
而天体辐射在大气内外的辐射流之比为$\dfrac{F_{\lambda}}{F_{0,\lambda}}=P_{\lambda}\left(z\right)$,因此大气内外星等差为
\begin{equation}
\Delta{}m=m_{\lambda}-m_{0,\lambda}=-2.5\lg\frac{F_{\lambda}}{F_{0,\lambda}}=-2.5M\left(z\right)\lg P_{\lambda}\left(0\right).
\end{equation}

\textbf{大气折射}:地球大气有一定的折射率,观测到的天体位置跟真实位置有偏折,改变天体辐射的方向。电离层中等离子体也会使天体辐射发生折射。

当天体辐射的波长大于等离子体的震荡波长时,会引起电离层的全反射而使天体辐射无法穿过大气到达地面。

\textbf{大气本身的辐射}:电离层中电子与离子复合产生荧光发射线。来源有$\ce{O,Na,O2,OH,H}$。除了极光,一般的天光发射与纬度没有强的依赖关系。另外,地球大气从地面直到$40\text{\textendash}\qty{60}{km}$高的区域,可假设为局部热平衡,发射黑体辐射,极大值在$\qty{10}{\mu m}$.因此地面可见光、红外、毫米波段观测都受影响。

\textbf{大气湍动}:由于大气中微弱但总是存在的热气流,使得星象在高倍放大下总是显得闪烁不定。影响光学成像的质量。

%\printbibliography
\end{document}